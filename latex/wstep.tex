Projekt obejmuje utworzenie robota w konfiguracji RPR $|--$. W raporcie zostały uwzględnione aspekty obliczeniowe takie jak kinematyka prosta, kinematyka odwrotna, kinematyka prędkości, równania Lagranga II rodzaju. 

W raporcie operujemy na wartościach symbolicznych takich jak np. $l_1$, jednak do obliczeń w języku Python\cite{Python} zostały użyte rzeczywiste wartości zgodnie z modelem w programie Inventor\cite{Inventor}.

Dodatkowo zostały określone takie rzeczy jak przestrzeń robocza manipulatora, tabele notacji Denavita-Hartenberga\cite{DH}, uproszczony model manipulatora na wykresie Robotics ToolBox\cite{Robotics_ToolBox}.

Projekt jest dostępny jako repozytorium na serwisie GitHub\cite{GitHub} pod linkiem  \url{https://github.com/maciejm2517/Projekt-Robot-RPR}

Ustawienia układu stron oraz bibliografii w języku \LaTeX zostały zainspirowane formatką Pana mgr inż. Bogdana Fabiańskiego\cite{Bogdan_Fabianski}, od którego otrzymaliśmy plik szablonowy na zajęciach Techniki Cyfrowej.