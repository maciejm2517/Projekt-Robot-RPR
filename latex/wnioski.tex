Budowa robota w praktyce różni się znacząco od obliczeń prowadzonych na kartce. Trzeba uwzględnić różne opory, mechanizmy, przeszkody w wykonywaniu modelu. Mogliśmy tego również doświadczyć na laboratoriach z tego przedmiotu, gdzie pracowaliśmy z prawdziwymi maszynami \textbf{KUKA} czy \textbf{STÄUBLI}


Nasz projekt również wymagał od nas uwzględnienia takich przeciwności jak rzeczywista konstrukcja połączeń czy sposoby blokady mechanizmów, aby robot mógł zostać złożony w rzeczywistości.


Możemy również zauważyć, że parametry takie jak przestrzeń robocza różnią się od obliczeń prowadzonych na kartce - teoretycznie przy naszej konfiguracji robota, przestrzenią roboczą powinien być dysk, jednak w rzeczywistości jest nią spłaszczony walec (bez środka). Wynika to z faktu że końcówka manipulatora nie jest punktem w przestrzeni, a ma swój rozstaw końcówek.