
\begin{figure}[H] 
\centering

$J_1=
\begin{bmatrix}
\overrightarrow{z_0}\times \overrightarrow{ (0_n-0_0) }\\
 \overrightarrow{z_0}\\
\end{bmatrix}  $
 
  \end{figure}


 \begin{figure}[H]
\centering

$z_0=
\begin{bmatrix}
0\\
0 \\
1\\
\end{bmatrix}  $
 
  \end{figure}

 \begin{figure}[H]
\centering

$0_0=
\begin{bmatrix}
0\\
0 \\
0\\
\end{bmatrix}  $
 
  \end{figure}


 \begin{figure}[H]
\centering

$O_n=
\begin{bmatrix}
l_3\times c_1 +c_1\times(l_2+d_2)\\
l_3 \times s_1 +s1\times (l_2+d_2) \\
l_1\\
\end{bmatrix}  $
 
  \end{figure}


 \begin{figure}[H]
\centering

$J_1=
\begin{bmatrix}
-l_3\times s_1-s_1\times(l_2+d_2)\\
 l_3\times c_1 +c_1\times(l_2+d_2)\\
 0\\
\end{bmatrix}  $
 
 \end{figure} 


 \begin{figure}[H]
\centering

$J_2=
\begin{bmatrix}
\overrightarrow{z_1}\\
0\\
0\\
0\\
\end{bmatrix}  
=
\begin{bmatrix}

c1\\
s1\\
0\\
0\\
0\\
0\\
\end{bmatrix}  
$
 
  \end{figure}


\begin{figure}[H]
\centering

$J_3=
\begin{bmatrix}
\overrightarrow{z_2}\times \overrightarrow{0_n-0_2}\\
\overrightarrow{z_2}\\
\end{bmatrix}  $
 
  \end{figure}

\begin{figure}[H] 
\centering

$\overrightarrow{0_n-0_2}=
\begin{bmatrix}
c_1\times(l_2+d_2)\\
s_1\times(l_2+d_2) \\
l_1\\
\end{bmatrix}  $
 
  \end{figure}

\begin{figure}[H]
\centering

$\overrightarrow{z_2}=
\begin{bmatrix}
c_1\\
s_1 \\
0\\
\end{bmatrix}  $
 
  \end{figure}


\begin{figure}[H]
\centering

$J_3=
\begin{bmatrix}
0\\
0 \\
0\\
c1\\
s1\\
0\\
\end{bmatrix}  $
 
  \end{figure}


\begin{figure}[H]
\centering

$J=
\begin{bmatrix}
-l_3\times s_1 - s_1\times(l_2+d_2) &c_1 &0\\
l_3\times c_1+c_1\times(l_2+d_2) & s_1 & 0 \\
0 & 0 & 0\\
0 & 0 & c1\\
0 & 0 & s1 \\
1 & 0 & 0\\
\end{bmatrix}  $
 
  \end{figure}

\begin{figure}[H]
\centering

$
\begin{bmatrix}
v\\
\omega \\
\end{bmatrix}
=
\begin{bmatrix}
-l_3\times s_1 - s_1\times(l_2+d_2) &c_1 &0\\
l_3\times c_1+c_1\times(l_2+d_2) & s_1 & 0 \\
0 & 0 & 0\\
0 & 0 & c1\\
0 & 0 & s1 \\
1 & 0 & 0\\
\end{bmatrix}  
\begin{bmatrix}
\dot{\theta_1} \\
\dot{d_2} \\
\dot{\theta_3}\\

\end{bmatrix}  $
\caption{Wektory prędkości linowej i kątowej końcówki względem układu zerowego} 
  \end{figure}